%!TEX TS-program = xelatex

\documentclass[russian]{beamer}

% Поддержка русского языка
\usepackage{cmap}	% поиск в PDF
\usepackage[T2A]{fontenc}		% кодировка
\usepackage[english, russian]{babel}	% локализация и переносы
\usepackage[utf8]{inputenc}	% кодировка исходного текста
\usepackage{mathtext}	% русские буквы в формулах
\usepackage{indentfirst}	% красная строка отбивка
\frenchspacing
\usepackage{fontspec}	% подгатавливает загрузку шрифтов Open Type, True Type и др.
\defaultfontfeatures{Ligatures={TeX}, Renderer=Basic}	%% свойство шрифтов по умолчанию
\setmainfont[Ligatures={TeX, Historic}]{Times New Roman}	%% задает основной шрифт документа
\setsansfont{Comic Sans MS}	%% задает шрифт без засечек
\setmonofont{Courier New}


% Переопределение греческих символов для русского языка
% ensuremath нужна для того, если ты не пишешь в $, то автоматически подставит $-ы
\renewcommand{\epsilon}{\ensuremath{\varepsilon}}
\renewcommand{\phi}{\ensuremath{\varphi}}
\renewcommand{\kappa}{\ensuremath{\varkappa}}
\renewcommand{\le}{\ensuremath{\leqslant}}
\renewcommand{\leq}{\ensuremath{\leqslant}}
\renewcommand{\ge}{\ensuremath{\geqslant}}
\renewcommand{\geq}{\ensuremath{\geqslant}}
\renewcommand{\emptyset}{\ensuremath{\varnothing}}

\usepackage{amssymb, amsfonts, amsmath, mathtext}
\usepackage{graphicx}

\usetheme{CambridgeUS}
%\usefonttheme{professionalfonts}
\usefonttheme[onlymath]{serif}

\title{Тема докалада}
\author{Ларина~\,Е.\,А}
\institute{Уральский Федеральный Университет}
\date{Екатеринбург, 2021}

\begin{document}

\begin{frame}
	\maketitle
	\setcounter{framenumber}{0}
\end{frame}

\begin{frame}
	\frametitle{Содержание}
	\tableofcontents
\end{frame}

\section{Формулы}
\begin{frame}
\frametitle{\insertsection}
\begin{block}{Темы}
Список тем и цветовой палитры для beamer можно найти на ресурсе:\newline
\href{https://hartwork.org/beamer-theme-matrix/}{https://hartwork.org/beamer-theme-matrix/}
\end{block}
Для формул возможен вариант использования \textsc{\textbackslash displaystyle}, чтобы формулы не сжимались и были хорошо восприняты глазом.\newline
Просто ввести эту команду в \textsc{\$} перед написанием формулы
$$
	x^2=0
$$
Также можно использовать оверлеи, анимацию и звук.\newline
Теперь вставим 2 формулы в ряд\newline
\begin{center}
\begin{tabular}{cc}
$x^2=0$ & $y=x^2$
\end{tabular}
\end{center}
\end{frame}

\section{Текст}
\begin{frame}
	\frametitle{\insertsection}
	Стандартные сокращения пишутся через команды: \textsc{т.\textbackslash:д.} и если существует предлог <<и>>, тогда через команду \textsc{и$\sim$ т.\textbackslash:д.}\newline
Чтобы правильно ставить тире, нужно писать так: \textsc{"---}. Здесь тройное тире.\newline
\textsf{Текст без засечек}\newline
\texttt{Моноширинный}\newline
\fontspec{Arial}{Изменили шрифт для этого предложения на ходу}\newline
\fontspec{Arial Narrow}{Еще другой взяли шрифт}
\end{frame}

\end{document}