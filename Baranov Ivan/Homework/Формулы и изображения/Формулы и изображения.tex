%!TEX TS-program = xelatex

% Этот шаблон документа разработан в 2014 году
% Данилом Фёдоровых (danil@fedorovykh.ru) 
% для использования в курсе 
% <<Документы и презентации в \LaTeX>>, записанном НИУ ВШЭ
% для Coursera.org: http://coursera.org/course/latex .
% Исходная версия шаблона --- 
% https://www.writelatex.com/coursera/latex/5.2.2

\documentclass[a4paper,14pt]{extarticle}

%%% Работа с русским языком
\usepackage[english,russian]{babel}   %% загружает пакет многоязыковой вёрстки
%\usepackage{fontspec}      %% подготавливает загрузку шрифтов Open Type, True Type и др.
%%% Подготавливает загрузку шрифтов Open Type, True Type и др.
\usepackage[euenc]{fontspec}
\defaultfontfeatures{Ligatures={TeX},Renderer=Basic}  %% свойства шрифтов по умолчанию
\setmainfont[Ligatures={TeX,Historic}]{Times New Roman} %% задаёт основной шрифт документа  %% Поставить потом CMU Serif
%\setsansfont{Comic Sans MS}                    %% задаёт шрифт без засечек
\setmonofont{Courier New}
\usepackage{indentfirst}
\frenchspacing

\renewcommand{\epsilon}{\ensuremath{\varepsilon}}
\renewcommand{\phi}{\ensuremath{\varphi}}
\renewcommand{\kappa}{\ensuremath{\varkappa}}
\renewcommand{\le}{\ensuremath{\leqslant}}
\renewcommand{\leq}{\ensuremath{\leqslant}}
\renewcommand{\ge}{\ensuremath{\geqslant}}
\renewcommand{\geq}{\ensuremath{\geqslant}}
\renewcommand{\emptyset}{\varnothing}

% Убираем нумерацию section
% \newcommand{\anonsection}[1]{\section*{#1}\addcontentsline{toc}{section}{#1}}

%%% Дополнительная работа с математикой
\usepackage{amsmath,amsfonts,amssymb,amsthm,mathtools} % AMS
\usepackage{icomma} % "Умная" запятая: $0,2$ --- число, $0, 2$ --- перечисление

%% Номера формул
%\mathtoolsset{showonlyrefs=true} % Показывать номера только у тех формул, на которые есть \eqref{} в тексте.
%\usepackage{leqno} % Нумерация формул слева

%% Свои команды
\DeclareMathOperator{\sgn}{\mathop{sgn}}

%% Перенос знаков в формулах (по Львовскому)
\newcommand*{\hm}[1]{#1\nobreak\discretionary{}
{\hbox{$\mathsurround=0pt #1$}}{}}

%%% Работа с картинками
\usepackage{graphicx}  % Для вставки рисунков
\graphicspath{{images/}{images2/}}  % папки с картинками
\setlength\fboxsep{3pt} % Отступ рамки \fbox{} от рисунка
\setlength\fboxrule{1pt} % Толщина линий рамки \fbox{}
\usepackage{wrapfig} % Обтекание рисунков текстом

%%% Подписи к картинкам и таблицам
\usepackage[tableposition=top]{caption}
\usepackage{subcaption}
\DeclareCaptionLabelFormat{gostfigure}{Рисунок #2}
\DeclareCaptionLabelFormat{gosttable}{Таблица #2}
\DeclareCaptionLabelSeparator{gost}{~---~}
\captionsetup{labelsep=gost}
\captionsetup[figure]{labelformat=gostfigure}
\captionsetup[table]{labelformat=gosttable}
\renewcommand{\thesubfigure}{\asbuk{subfigure}}

%%% Работа с таблицами
\usepackage{array,tabularx,tabulary,booktabs} % Дополнительная работа с таблицами
\usepackage{longtable}  % Длинные таблицы
\usepackage{multirow} % Слияние строк в таблице

%%% Теоремы
\theoremstyle{plain} % Это стиль по умолчанию, его можно не переопределять.
\newtheorem{theorem}{Теорема}[section]
\newtheorem{proposition}[theorem]{Утверждение}
 
\theoremstyle{definition} % "Определение"
\newtheorem{corollary}{Следствие}[theorem]
\newtheorem{problem}{Задача}[section]
 
\theoremstyle{remark} % "Примечание"
\newtheorem*{nonum}{Решение}

%%% Программирование
\usepackage{etoolbox} % логические операторы


%%% Страница
\usepackage{extsizes} % Возможность сделать 14-й шрифт
\usepackage{geometry} % Простой способ задавать поля
	\geometry{top=20mm}
	\geometry{bottom=20mm}
	\geometry{left=30mm}
	\geometry{right=15mm}
 %
%\usepackage{fancyhdr} % Колонтитулы
% 	\pagestyle{fancy}
 	%\renewcommand{\headrulewidth}{0pt}  % Толщина линейки, отчеркивающей верхний колонтитул
% 	\lfoot{Нижний левый}
% 	\rfoot{Нижний правый}
% 	\rhead{Верхний правый}
% 	\chead{Верхний в центре}
% 	\lhead{Верхний левый}
%	\cfoot{Нижний в центре} % По умолчанию здесь номер страницы

\usepackage{setspace} % Интерлиньяж
    \onehalfspacing % Интерлиньяж 1.5
%\doublespacing % Интерлиньяж 2
%\singlespacing % Интерлиньяж 1

\usepackage{lastpage} % Узнать, сколько всего страниц в документе.

\usepackage{soul} % Модификаторы начертания


\usepackage{hyperref}
\usepackage[usenames,dvipsnames,svgnames,table,rgb]{xcolor}
\hypersetup{				% Гиперссылки
    unicode=true,           % русские буквы в раздела PDF
    pdftitle={Заголовок},   % Заголовок
    pdfauthor={Автор},      % Автор
    pdfsubject={Тема},      % Тема
    pdfcreator={Создатель}, % Создатель
    pdfproducer={Производитель}, % Производитель
    pdfkeywords={keyword1} {key2} {key3}, % Ключевые слова
    colorlinks=true,       	% false: ссылки в рамках; true: цветные ссылки
    linkcolor=black,          % внутренние ссылки
    citecolor=black,        % на библиографию
    filecolor=black,      % на файлы
    urlcolor=black           % на URL
}

\usepackage{csquotes} % Еще инструменты для ссылок

%\usepackage[style=authoryear,maxcitenames=2,backend=biber,sorting=nty]{biblatex}

\usepackage{multicol} % Несколько колонок

\usepackage{tikz} % Работа с графикой
\usepackage{pgfplots}
% \usepackage{pgfplotstable}
\pgfplotsset{compat=1.16}
\usepackage{pdfpages} % include other pdf file
\newcommand{\RNumb}[1]{\uppercase\expandafter{\romannumeral #1\relax}}

%%% Управление нумерацией элементов
\usepackage{chngcntr}
\counterwithin{figure}{section}
\counterwithin{equation}{section}
% \counterwithin{equation}{subsection}


%%% Списки
\usepackage{enumitem}
\makeatletter
    \AddEnumerateCounter{\asbuk}{\@asbuk}{м)}
\makeatother
\setlist{nolistsep}
\renewcommand{\labelitemi}{-}

%%% Первый уровень нумерации -- буквы, второй -- цифры 
\renewcommand{\labelenumi}{\asbuk{enumi})}
\renewcommand{\labelenumii}{\arabic{enumii})}

\usepackage[titles]{tocloft}
\renewcommand{\cfttoctitlefont}{\normalfont\Large\bfseries\uppercase}
\renewcommand{\cftsecleader}{\cftdotfill{\cftdotsep}}
\newcommand{\likesectionheading}[1]{
  \section*{\centering{#1}}
}

\newcommand{\likesection}[1]{
  \likesectionheading{#1}
  \addcontentsline{toc}{section}{{#1}}
}

\author{Баранов Иван}
\title{Диплом}
\date{\today}


%\usepackage{cancel}
%\usepackage{textcomp}
%\usepackage{graphicx}
%\usepackage{wrapfig}
%\usepackage{array}
%\usepackage{caption}
%\linespread{1.3}
%\setlength{\parindent}{5ex}

\begin{document}

\begin{center}
	Формулы
\end{center}
Формулы разделяются на внутри текста и <<выключенные>> (то есть выделенные в отдельную строку). \newline
Если используете в формуле десятичные дроби, в которых дробная часть отделена от целой с помощью запятой, то эту запятую следует взять в фигурные скобки (в противном случае после нее будет оставлен небольшой дополнительный пробел, что нежелательно).\newline
\textbf{Все примеры использования формул смотри в исходном коде!}\newline
С фигурными скобками
$$
	\pi\approx 3{,}14
$$
Без фигурных скобок
$$
	\pi\approx 3,14
$$

Для скобок в выражениях удобно использовать конструкцию \textsc{\textbackslash left} перед открывающейся скобокой и \textsc{\textbackslash right} перед закрывающейся скобкой. \newline

Использование команды \textsc{\textbackslash mid} определяет вертикальную черточку, рассматриваемую как знак бинарного отношения. Пример использования
\begin{equation*}
	M=\{\,x\in A\mid x>0\, \}
\end{equation*}

При записи отображений нужно использовать не двоеточие, а команду \textsc{\textbackslash colon}
\begin{equation*}
	f\colon X\to Y
\end{equation*}

Так как в стандартном комплекте Latex'a нет обозначений для отечественной литературы тригонометрических функций, то требуется подключить пакет \textsc{amsmath}. После чего добавить в преамбулу такую строчку
\begin{center}
	\textsc{\textbackslash DeclareMathOperator\{\textbackslash tg\}\{tg\}}
\end{center}

НУМЕРАЦИЯ ФОРМУЛ

Для того, чтобы нумеровать формулы, надо писать формулы как окружение \textsc{equation} и использовать команду \textsc{\textbackslash label}. После этого команда \textbf{\textbackslash ref} будет генерировать номер формулы (без скобок). Чтобы сразу использовать скобки, достаточно подключить пакет \textsc{amsmath} и использовать команду \textsc{\textbackslash eqref}. Также можно использовать команжу \textsc{\textbackslash pageref}, тогда на печати получится не номер формулы, а номер страницы, на которую попала эта формула.\newline
Можно не автоматически генерировать нумерацию формул, а делать это самостоятельно. Для этого при написании формул в двойных доларах нужно использовать команду \textsc{\textbackslash eqno (1.1)}

ПЕРЕНОСЫ В ФОРМУЛАХ

При необходимости можно переносить часть внутритекстовой формулы на другую строчку. Такие переносы возможны после знаков <<бинарных отношений>> или <<бинарных операций>>. Для этого можно использовать одну из двух преамбул:\newline
\textbf{\textbackslash binoppenalty=10000} - запретит все разрывы строк после знаков бинарных операций\newline
\textbf{\textbackslash relpenalty=10000} - запретит все разрывы строк после знаков бинарных отношений, и при этом помех верстке абзаца будет меньше, чем при заключении всей формулы в фигурные скобки.

ВКЛЮЧЕНИЕ ТЕКСТА В ФОРМУЛЫ

В математическую формулу можно включить фрагмент обычного текста с помощью команды \textsc{\textbackslash mbox}. В этой команде используется сразу команда \textsc{\textbackslash qquad}

ФОРМУЛА В РАМОЧКЕ

Если подключить пакет \textbf{amsmath}, то формулу можно взять в рамочку с помощью команды \textsc{\textbackslash boxed}
\begin{equation*}
	\boxed{x=2}
\end{equation*}

Использование команды \textsc{\textbackslash substack} пакета amsmath
\begin{equation*}
	\sum_{\substack{i\in[0;n]\\j\in [0,m]}} a_{ij}
\end{equation*}

МНОГОСТРОЧНЫЕ ВЫКЛЮЧНЫЕ ФОРМУЛЫ

1. Использовать окружение \textsc{multiline}\newline
2. Использовать окружение \textsc{gather}\newline
3. Если нужно напечатать один или несколько выровненных столбцов формул, то нужно использовать окружение \textsc{align}
\begin{align*}
	7\times 9 & =63 & 63:9&=7\\
	9\times10 & =90 & 90:10& =9
\end{align*}
Использование текста в окружении \textit{cases}
\begin{equation*}
	|x|=
	\begin{cases}
		x,&\text{если $x>0$;}\\
		0,&\text{если $x=0$;}\\
		-x,&\text{если $x<0$;}
	\end{cases}
\end{equation*}

СМЕНА ШРИФТОВ В ФОРМУЛЕ

По умолчанию все латинские буквы в формулах набираются курсивом.\newline
Для того, чтобы набирать жирным шрифтом, надо воспользоваться командой \textsc{\textbackslash mathbf}
\begin{equation*}
	\mathbf P^n
\end{equation*}
Команда \textsc{\textbackslash mathcal}, вызывающая <<каллигрофический>> шрифт, можно применять только к прописным латинским буквам.

Если подключить стилевой пакет \textsc{amsfonts} или \textsc{amssymb}

\begin{center}
	Изображения/графика
\end{center}

ПЛАВАЮЩИЕ ИЛЛЮСТРАЦИИ

Чтобы разместить в тексте иллюстрацию, удобно воспользоваться окружением \textit{figure}. Команда \textbf{\textbackslash caption} позволяет сделать подрисуночную подпись. Также есть команда \textbf{\textbackslash label}.

Окружение \textit{figure} может принимать аргумент в квадратных скобках:\newline
1. \textbf{t} - разместить иллюстрацию в верхней части страницы.\newline
2. \textbf{b} - разместить иллюстрацию в нижней части страницы.\newline
3. \textbf{p} - разместить иллюстрацию на отдельной странице, целиком состоящей из <<плавающих>> иллюстраций (или таблиц).\newline
4. \textbf{h} - разместить иллюстрацию прямо там, где она встетилась в исходном тексте, не перенося ее никуда.\newline
Можно в этот аргумент задавать сразу несколько параметров: \textit{\textbackslash begin\{figure\}[tbp]}

РИСУНКИ В ОБОРКУ

Окружения figure и table определяют иллюстрацию и таблицы, простирающие на всю ширину текста. Для того, чтобы сдлеать иллюстрацию с обтеканием текста, необходимо использовать пакет \textit{wrapfig}. Тогда, чтобы задать изображение, необходимо воспользоваться окружением \textit{wrapfigure}.\newline
Окружение wrapfigure имеет два обязательных аргумента. Первый из них указывает, как должен расположен рисунок относительно текста, а второй - ширину рисунка.\newline
Также окружение wrapfigure принимает необязательный аргумент, ставящийся перед первым обязательным, указывающий количество укороченных строк: \textit{\textbackslash \{wrapfigure\}[14]\{o\}\{60pt\}}. Имейте в виду, что при таких расчетах любая выключная формула считается за три строки.

\end{document}